\documentclass[11pt,a4paper]{article}
\usepackage[utf8]{inputenc}
\usepackage[french]{babel}
\usepackage[margin=1in]{geometry}
\usepackage{booktabs}
\usepackage{enumitem}
\usepackage{xcolor}
\usepackage{titlesec}
\usepackage{hyperref}
\usepackage{amsmath}

\definecolor{roogoprimary}{HTML}{E11D48} % Adjust to Roogo's actual primary color

\titleformat{\section}{\color{roogoprimary}\normalfont\Large\bfseries}{\thesection}{1em}{}
\titleformat{\subsection}{\normalfont\large\bfseries}{\thesubsection}{1em}{}

\title{\textbf{Stratégie Commerciale et Opérationnelle Roogo}}
\author{Direction Stratégique Roogo}
\date{\today}

\begin{document}

\maketitle

\section{Introduction}
Ce document récapitule les décisions stratégiques et techniques prises pour l'évolution de la plateforme Roogo au Burkina Faso (XOF). L'objectif est de supprimer les barrières à l'entrée pour les chercheurs de biens tout en garantissant une rentabilité via un modèle de service premium pour les propriétaires.

\section{Modèle de Revenus (Propriétaires)}
Le coût de publication est basé sur un frais fixe par pack plus une commission sur la valeur du bien pour assurer une différenciation claire des prix.
\medskip

\textbf{Formule de Tarification :} \\
\text{Prix} = \text{Frais Fixe du Pack} + 5\% \text{ du loyer mensuel}

\subsection{Packs de Publication}
\begin{table}[h]
\centering
\begin{tabular}{lccccr}
\toprule
\textbf{Pack} & \textbf{Photos Pro} & \textbf{Slots App.} & \textbf{Vidéo} & \textbf{Open House} & \textbf{Frais Fixe} \\
\midrule
Essentiel & 8 & 25 & Non & 1 & 15 000 XOF \\
Standard & 8 & 50 & Oui & 2 & 25 000 XOF \\
Premium & 15 & 100 & Oui & 3 & 45 000 XOF \\
\bottomrule
\end{tabular}
\caption{Structure des packs (Base fixe + 5\% loyer)}
\end{table}

\section{Expérience Chercheur (Gratuité)}
Pour maximiser l'adoption et l'engagement des utilisateurs cherchant un logement :
\begin{itemize}[label=\checkmark]
    \item \textbf{Zéro Frais de Visite :} Les utilisateurs ne paient plus pour visiter les propriétés.
    \item \textbf{Modèle Open House :} Les visites sont groupées par sessions de 1h gérées par le staff Roogo.
    \item \textbf{Réservation Native :} Possibilité de réserver son créneau directement dans l'application après avoir postulé.
\end{itemize}

\section{Gestion du Cycle de Vie des Annonces}
Le système utilise l'urgence pour motiver les chercheurs et assurer une rotation rapide des biens.
\begin{enumerate}
    \item \textbf{Phase d'Attente :} L'annonce est créée avec des photos temporaires du propriétaire (\texttt{en\_attente}).
    \item \textbf{Mise en Ligne :} Le staff Roogo remplace les photos par des clichés professionnels (\texttt{en\_ligne}). La fenêtre Early Bird de 48h démarre.
    \item \textbf{Verrouillage (optionnel) :} Si un utilisateur réserve pendant la fenêtre Early Bird, l'annonce passe en statut \texttt{locked} et est retirée du marché.
    \item \textbf{Candidatures Ouvertes :} Après 48h sans réservation, le processus classique de candidature s'ouvre.
    \item \textbf{Clôture Automatique :} Dès que le quota de slots d'applications du pack est atteint, l'annonce passe en statut \texttt{closing}.
    \item \textbf{Expiration :} L'annonce reste visible 3 jours après avoir été complète pour générer de l'intérêt futur, puis passe en \texttt{expired}.
\end{enumerate}

\subsection{Diagramme des Statuts}
\begin{verbatim}
en_attente → en_ligne → [locked]     → (hors plateforme)
                     ↓
                  (48h passées)
                     ↓
                  candidatures → closing → expired
\end{verbatim}

\section{Réservation Anticipée (Early Bird)}
Ce modèle permet aux chercheurs sérieux de sécuriser un bien immédiatement après sa mise en ligne, avant que les candidatures classiques ne soient ouvertes.

\subsection{Principe}
Durant les \textbf{48 premières heures} suivant la mise en ligne d'une annonce, un utilisateur peut payer des frais de réservation pour \textbf{verrouiller} le bien. Une fois verrouillé :
\begin{itemize}
    \item Le bien est retiré du marché (statut \texttt{locked})
    \item Les autres utilisateurs ne peuvent plus postuler
    \item Le propriétaire est mis en contact direct avec l'utilisateur réservant
    \item Le paiement du loyer et la signature du bail se font hors plateforme
\end{itemize}

\subsection{Fenêtre Early Bird (48h)}
\begin{enumerate}
    \item \textbf{H0 - H48 :} Bouton ``Réserver'' visible avec compte à rebours
    \item \textbf{H48+ :} Bouton ``Réserver'' disparaît, processus de candidature classique s'ouvre
\end{enumerate}

Un compte à rebours dynamique dans l'application crée de l'urgence et encourage les utilisateurs à agir rapidement.

\subsection{Tarification Early Bird}
Les frais de réservation sont calculés en pourcentage du loyer mensuel :

\medskip
\textbf{Formule :} \\
\text{Frais de Réservation} = 10\% \times \text{Loyer Mensuel}

\medskip
Avec un \textbf{minimum de 10 000 XOF} pour garantir la rentabilité sur les petits loyers.

\begin{table}[h]
\centering
\begin{tabular}{lcr}
\toprule
\textbf{Loyer Mensuel} & \textbf{Calcul} & \textbf{Frais Early Bird} \\
\midrule
50 000 XOF & 10\% = 5K (< min) & 10 000 XOF \\
100 000 XOF & 10\% & 10 000 XOF \\
150 000 XOF & 10\% & 15 000 XOF \\
250 000 XOF & 10\% & 25 000 XOF \\
500 000 XOF & 10\% & 50 000 XOF \\
\bottomrule
\end{tabular}
\caption{Exemples de tarification Early Bird (10\% du loyer, min 10K)}
\end{table}

\subsection{Politique de Remboursement}
\begin{itemize}
    \item \textbf{Annulation par le locataire :} Aucun remboursement (filtre les indécis)
    \item \textbf{Rejet par le propriétaire (sous 48h) :} Remboursement de 50\%
    \item \textbf{Défaut de contact du propriétaire (72h+) :} Remboursement intégral
\end{itemize}

\subsection{Avantages Stratégiques}
\begin{itemize}
    \item \textbf{Nouveau flux de revenus :} Monétisation côté demande sans toucher au modèle gratuit de base
    \item \textbf{Rotation accélérée :} Les biens se louent plus vite, satisfaction propriétaire augmentée
    \item \textbf{Engagement utilisateur :} Le compte à rebours incite à l'ouverture quotidienne de l'app
    \item \textbf{Filtre qualité :} Seuls les locataires sérieux paient, réduisant le travail de tri pour les propriétaires
\end{itemize}

\section{Options de Personnalisation (Add-ons)}
\begin{itemize}
    \item +25 Candidats : 7 500 XOF
    \item +5 Photos professionnelles : 10 000 XOF
    \item Vidéo Professionnelle (si non incluse) : 10 000 XOF
    \item Environnement 3D : 25 000 XOF
    \item Boost "A la Une" (7 jours) : 7 000 XOF
    \item +1 Session Open House : 3 000 XOF
\end{itemize}

\section{Flux Opérationnel}
\begin{itemize}
    \item Le propriétaire paie et publie via l'application mobile.
    \item Le staff Roogo est notifié pour planifier la séance photo.
    \item Les photos professionnelles valident la mise en ligne définitive.
    \item Le staff définit les créneaux de visite via la console d'administration web.
\end{itemize}

\end{document}
